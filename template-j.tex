%% 和文論文用のテンプレート
%%%%%%%%%%%%%%%%%%%%%%%%%%%%%%%%%%%%%%%%%%%%%%%%%%%%%%%%%%%%%%%%%%%%%%%%%%%%
%% 1. 和文原稿
% \documentclass[originalpaper]{jsaiart}     % 原著論文 Original Paper
%\documentclass[blindreview]{jsaiart}      % 査読用
%
% \documentclass[shortpaper]{jsaiart}       % 速報論文 Short Paper
\documentclass[exploratorypaper]{jsaiart} % 萌芽論文 Exploratory Research Paper
% \documentclass[Specialissue]{jsaiart}     % 特集 Special Issue
% \documentclass[specialissue]{jsaiart}     % 小特集 Special Issue
% \documentclass[interimreport]{jsaiart}    % 報告 An Interim Report
% \documentclass[surveypaper]{jsaiart}      % 解説 Survey Paper
% \documentclass[aimap]{jsaiart}            % AIマップ AI map
% \documentclass[specialpaper]{jsaiart}     % 特集論文 Special Paper
% \documentclass[invitedpaper]{jsaiart}     % 招待論文 Invited Paper
%

%\usepackage{graphics}
\usepackage[dvipdfmx]{graphicx}

%% ページ番号の指定,掲載時に学会の方で決定します.
% \setcounter{page}{1}
% \setcounter{volpage}{1}


%%% amsmathパッケージの注意点 %%%%%%%%%%%%%%%%%%%%%%%%%%%%%%%%%%%%%%%
% \usepackpage{amsmath}
% 数式番号の参照は \ref ではなく,\eqref を用いること
% documentclass のオプションに fleqnを指定すること
% 例: \documentclass[technicalpaper,fleqn]{jsaiart}

\Vol{12}
\No{1}
\jtitle{一般化のためのネットワークプログラミング}
% \jtitle[柱用和文タイトル]{和文タイトル}
\jsubtitle{論理的推論によるビットNOT演算プログラムの獲得}
\etitle{Network Programming for Generalization (NP4G)}
\esubtitle{Getting a bitwise NOT operator's program by logical inference}

% \manyauthor % 著者が3名以下の場合はこの行を消すこと

%%% 著者名の注意点 %%%%%%%%%%%%%%%%%%%%%%%%%%%%%%%%%%%%%%%%%%%%%%%%%%%
% 所属先が同じ著者が連続する場合,その中の先頭の著者のみ \affiliation
% を用い,残りの所属先には \sameaffiliation を使う
% ただし,所属先が同じでも連続していない場合は \affliation を使う
% 名前が長い場合は \name の代りに \longname を使う

\author{%
 \name{原}{匠一郎}{Shoichiro Hara}
 \affiliation{名古屋市立大学}%
     {Nagoya City University}%
     {s.hara@nsc.nagoya-cu.ac.jp}
\and
 \name{渡邊}{裕司}{Yuji Watanabe}
 \sameaffiliation{yuji@nsc.nagoya-cu.ac.jp}
}

\begin{keyword}
logical inference, understanding causality, NP4G, GA, GP, GNP
\end{keyword}

\begin{summary}
「ショートノート」は 200 ワード,それ以外は200~500 ワード
以内の英文でsummaryを記す
\end{summary}

\begin{document}
\maketitle

\section{はじめに}
引用の例\cite{latex,texbook}論理的推論

人工知能の分野において,情報の構造を表現する方法がいくつも考案されてきたが,それらの最終目的でもある,論理的思考に基づいた自然言語の理解や,論理的推論による複雑な問題の解決は未だ困難である.

機械学習の発展における次の段階として,概念関係の理解は非常に重要である.

本研究では,論理的推論を行うことができる一般化のためのネットワークプログラミング(NP4G: Network Programming for Generalization)を提案する.
\section{関連研究}
\subsection{GP / GNP}
類似研究として,GNP

NP4GはGNPから遺伝的操作であることの意味をなくし(制限をなくし),一般化のための手法であるという意味を付け加えたものであるといえる.

終端ノード

GP/GNPでは各ノードを単純な処理をする最小単位と考え,それらを木構造状またはネットワーク状に接続することによりプログラムを構築する.

ネットワークにする理由
\subsection{ニューラルネットワーク(自動プログラミング)}
ニューラルネットワークの手法では,確かに個々のデータの特徴の共通点を拾い出すことができるため,その点においては一般化ができると言えるが,その手続きはブラックボックス化されており,論理的な推論における一般化とは言えない.ニューラルネットワークでは学習に基づき推論を行うが,その推論の思考プロセスは見えにくく,論理的でない.

ニューラルネットワークでは原因と効果や,なぜ関連性や相関があるのかを解釈できない.特に,創造や計画,推論を伴うタスクは得意としない.これと同様に,AIが学習を一般化することに限界がある.一般化の欠如は大きい問題である.

ニューラルネットワークでは基本ネットワークの構成は変えられず,重みを変化させることで学習するが,NP4Gではネットワークの構成を変えることで,学習する.


GPT3などの手法を用いた自動プログラミング


\subsection{本研究の目的}


\section{提案手法}
NP4G (Network Programming for Generalization)とは,教師データに基づき,ネットワークで表現されるプログラムを自動生成することで,論理的推論における一般化を行う手法である.

教師データ(入力と出力のペア)に合致するネットワークを探索する.

簡単な機能を持った複数のノード素子をネットワーク状に接続し,プログラムを記述することのできるモデル

入力をある規則に従って変換し出力することのできるモデル

ノード間の接続を教師データに合わせて変更することによって,教師データに合致するような処理のできるプログラムを得る.

ネットワークを生成する.

NP4G (Network Programming for Generalization)は論理的推論における一般化のモデル生成を目的とする.

論理的推論における一般化に関しては初の試みであり,
\subsection{自動定義関数(ADFs)}

\subsection{段階的学習}
\subsection{提案手法を使う利点}

\section{実験}
\subsection{設定}
NP4Gを用いて,ビットNOT演算プログラムを獲得する実験を行った.

\subsection{与えられる関数}
\subsubsection{div関数}
\begin{figure*}[t]
    \begin{center}
        %\includegraphics[width=90mm]{ab0ex-mtphu.eps}
        \includegraphics[width=50mm]{div.png}
        %\epsfile{file=ab0ex-mtphu.eps,width=90mm}
    \end{center}
    \capwidth=50mm %
    \caption{図の説明文... }
\end{figure*}

\subsubsection{sum関数}
\subsubsection{equal関数}
\subsubsection{制御ゲート関数}
\subsection{結果}
\subsubsection{得られたネットワーク}

\begin{figure*}[t]
    \begin{center}
        %\includegraphics[width=90mm]{ab0ex-mtphu.eps}
        \includegraphics[width=120mm]{network.jpg}
        %\epsfile{file=ab0ex-mtphu.eps,width=90mm}
    \end{center}
    \capwidth=90mm %
    \caption{図の説明文... }
\end{figure*}

\section{おわりに}
\subsubsection{今後の課題}
チューリング完全

\bibliography{btxsample}
\bibliographystyle{jsai}

\appendix

\section{付録のタイトル1}

付録の本文1

\section{付録のタイトル2}

付録の本文2

% 著者の姓と名の間は半角スペースで区切る
% 略歴は200字以内
\begin{biography}
\profile{s}{原 匠一郎}{2018年3月名城大学理工学部電気電子工学科卒業}
\profile{n}{渡邊 裕司}{著者2の略歴}
%\profile*{m}{著者姓 名}{前掲\kern-.5zw (Vol.X,No.Y,p.Z)\kern-.5zw 参照.}
\end{biography}

\end{document}
